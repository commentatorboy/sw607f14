\section*{Reading guide}\addcontentsline{toc}{chapter}{Reading guide}

To understand this report in its entirety, some basic programming knowledge is required, although most parts can be read without any special prerequisites.

When reading this report, it is important to note that the report and the development described is only one part of a multi-project split across several groups. Some groups are mentioned in the report by functionality rather than name as it is easier to read. These groups are as follows:

\begin{itemize}
\item sw602f14: Livshistorier (LifeStories)
\item sw603f14: GKomponenter (GComponents)
\item sw607f14: Sekvens (Sequence)
\item sw609f14: Kategoriværktøjet (CategoryLib)
\item sw611f13: PictoSearch
\item sw614f14: OasisLib
\end{itemize}

The overall solution is called GIRAF and will be addressed as such.

Whenever "we" are mentioned in the report, this refers to the Sekvens group, sw607f14.

Because Sekvens originally had certain name conventions throughout the codebase, these has been preserved in the report. These are the relevant terms to know:

\begin{itemize}
\item Guardian: An adult who is a caretaker for autistics. This can both represent parents and guardians.
\item Child: An autistic person.
\end{itemize}