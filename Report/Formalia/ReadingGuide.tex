\section*{Reading guide}\addcontentsline{toc}{chapter}{Reading guide}

To understand this report in its entirety, some basic programming knowledge is required, although most parts can be read without any special prerequisites.

The report is built based on the the product backlog following the scrum model used (See \ref{spr1_projmana}). This means that the report for the most part is split into sprints and issues that have been worked on. The full product backlog can be seen in \ref{app:productbacklog}.

When reading this report, it is important to note that the report and the development described is only one part of a multi-project split across several groups. Some groups are mentioned in the report by functionality rather than name as it is easier to read. These groups are as follows:

\paragraph{sw601f14: Requirements}
Responsible for customer contact and obtaining requirements for all groups.
\paragraph{sw602f14: Livshistorier}
Developers for the Livshistorier application, used to create stories from pictograms.
\paragraph{sw603f14: GComponents}
Responsible for creating graphical components that all groups can access and use
\paragraph{sw605f14: GIRAF}
Developers for the launcher front-end which serves as a home screen for applications within the GIRAF multi-project.
\paragraph{sw607f14: Sekvens (Sequence) and Sequenceviewer}
Our own group, responsible for the Sekvens application and later in the project also SequenceViewer.
\paragraph{sw609f14: Kategoriværktøjet}
Developers for a tool applications where users can sort pictograms into categories.
\paragraph{sw610f14: iOS}
Developers for an application similar to that of Sekvens, but for the iOS platform.
\paragraph{sw611f14: PictoSearch}
Developers for a search program, designed to find pictograms in the database.
\paragraph{sw614f14: OasisLib}
Responsible for the local database and models.
\paragraph{sw615f14: PiktoOplæser}
Developers of a communication tool designed to read sequences of pictograms for the user.\\
\\
The overall solution is called GIRAF, but will be addressed as the GIRAF multi-project.

Whenever "we" are mentioned in the report, this refers to the Sekvens group, sw607f14.

Because Sekvens originally had certain name conventions throughout the codebase, these has been preserved in the report. These are the relevant terms to know:

\begin{itemize}
\item Guardian: An adult who is a caretaker for autistics. This can both represent parents and guardians.
\item Child: A person suffering from autistic spectrum disorder.
\end{itemize}