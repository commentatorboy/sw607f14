\subsection{Test Results}\label{subsec:test_results}
Both test subjects had never used the application before, although knowing about the project in general. This means we got an insight as to how intuitive and user-friendly the program is for new users.

Beside the test we actually got a request when finishing up. One test subject would like to be able to send sequences to email. This originated in the Livshistorier application being able to do this. This is seperated from the test because it is not an issue with the program but a request for new functionality.

Under the test, several issues were highlighted. These has been categorized within the following categories:

\begin{itemize}
\item C: Cosmetic/minor issue such as a wrong icon or a bug with little interference.
\item S: Severe difficulty completing the assigned task. Takes 30 seconds or more to find a functionality.
\item F: Critical error. User is not able to complete the task and is unable to proceed without assistance.
\end{itemize}

These are the issues found in Sekvens:

\paragraph{C: Misleading icon}
A visual issue immediately recognized by the second test subject was that the icon for the exit button was a red X. We were informed that this usually means something is canceled, and suggested to find a different icon. As this is an icon from GIRAF\_Components that several groups use, this would be preferred changed from there. An alternative icon could be a man going out through a door.

\paragraph{C: Choosing multiple pictograms at the same time}
One of the test subjects attempted to insert a nested sequence instead of choosing multiple pictograms. This could possibly be solved easily by renaming "pictogram" to "pictogrammer" (pictograms) in the add dialog. This would highlight that it is possible to insert more than one pictogram.

\paragraph{C: Identifying if settings/copy is saved}
One test subject was unable to see if the sequence she had copied and the settings she had altered were actually saved. This could easily be solved by displaying a dialog box informing of the result. The test subject however correctly assumed changes were indeed saved.

\paragraph{S: Locating Nested/Choice feature}
Both test subjects had issues locating these features and generally took a long time finding them. Both test subjects however stated that after using it once, it was quite intuitive. An alternative could be to discard the dialog and instead add a seperate button for each feature. This would however also create some cluttering as there are already 6 buttons in the activity.

One test subject had a hard time locating the choice button because she mistook it for a settings or special menu. A possible fix could be to find a better naming for a choice in a sequence.

It was suggested by one test subject that the placeholder for a choice should be a question mark similar to Livshistorier.

\paragraph{S: Identifying a choice or nested sequence}
Both test subjects were unable to directly identify which pictogram in a sequence was a choice or a nested sequence. One of them were able to find it by trial-and-error while the other simply reached the conclusion that it could not be seen. This was however expected as identifying placeholders had not been implemented at this point.

\paragraph{S: Locating the copy/delete button}
One test subject had difficulties locating the copy and delete buttons. Once finding them however, she had no idea why she did not see them in the first place, but emphasized that it was very intuitive. From observation, she was not in doubt she had found the right buttons once seeing them. Rather than attempting to fix this issue, we attribute it to be an unfortunate oversight by the test subject. Having more test subjects could help identify if it really is an issue.