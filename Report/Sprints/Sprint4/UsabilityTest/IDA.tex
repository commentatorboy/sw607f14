\subsection{IDA - Instant Data Analysis}\label{subsec:IDA}
For the usability test it was decided to roughly use the model of Instant Data Analysis (IDA). IDA has the benefit of being very time-efficient while still highlighting many issues \cite{IDA}.

The idea with IDA is to have a test subject perform given tasks on the software, and log everything at the same time. An alternative approach would be to record everything, transcribe it and analyze it later on. This is however very time expensive and exactly what is avoided here.
Instead, an analysis session is scheduled immediately after the test to create a list of discorvered issues.

IDA normally requires four people with each their unique role:

\paragraph{Test subject} A person who attempts to perform tasks on the software to test. Is asked to think aloud to clarify choices made when using the software, providing the feedback needed.

\paragraph{Test monitor} Interacts with the test subject in terms of introduction and informing of the tasks the test subject should perform. Sits next to the test subject, but does not help with the tasks. Can ask questions for clarification if the test subject is not clear on the basis of an action.

\paragraph{Data logger} Logs any noteworthy behavior from another room while the test subject performs the assigned tasks to the software. The test subject is usually monitored through a camera.

\paragraph{Facilitator} Manages the analysis session and helps identify and categorize issues.
