\begin{longtable} { | c | p{12cm} | c | } 
\hline
	ID 	&	Issues	&		 Es. hours \\\hline
	57	&	Add settings	&	16 hours	\\\hline
	60	&	Settings Intents & \\\hline
\caption{Issue ID 57 and 60}
\label{tab:spr4_addsettings}
\end{longtable}

To manage customer requests to display sequences differently for children, a settingsActivity was created with two options: The orientation of the screen and how many pictograms to display at a time.

The database had no support for saving settings. This forced us to implement the settings using Android's \ct{sharedPreferences} which is able to store preferences between user sessions \cite{sharedpreferences}. This means that while the preferences are persistant on the tablet they were altered, it is not synchronized with the database at any point.

The settings are saved when the guardian closes the settingsActivity. They are uniquely saved for each child as the key used to save/fetch the settings includes the ID of the child, which from the database should be guaranteed unique. When needed, they can be loaded using Android's \ct{getSharedPreferences}. This is done when calling SequenceViewer, and the preferences are sent through an \ct{intent}.