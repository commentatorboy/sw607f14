\section{Sprint Evaluation}
During the third sprint, a group member was forced to leave the group due to circumstances not within our control. This meant that from that point forward we had 1/4 hours less to work with. At the end of the sprint, SequenceViewer was set up as a very basic prototype, focus being foremost that it was functional. Sekvens had some major refactors which greatly helped readability. This was found to be needed, given the amount of code added to the application. From the user-side it became easier to understand, as helpful dialogs were added in some places.
Near the end of the sprint it was announced that the database was ready to handle sequences, but at this point it was too late to change anything for sprint 3.
We would like to have done usability tests in sprint 3, but according to the requirements group, this was not possible. This was unfortunate, because delaying it to sprint 4 meant that we had less time to correct any potential issues found under the usability test. 