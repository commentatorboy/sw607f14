\begin{longtable} { | c | p{12cm} | c | } 
\hline
	ID 	&	Issues	&		 Es. hours \\\hline
	50 	&	Sequenceviewer: an arrow highlighting method	&	16 hours \\\hline
\caption{Issue ID 50}
\label{tab:spr3_SVarrowhighlight}
\end{longtable}

As stated in appendix \ref{app:reqgroup1}, it was a requirement that the users can mark how far they are in a sequence. This has been implemented by setting up a picture of an arrow, which are placed above the first pictogram in a sequence.

In landscape$\_$mode.xml, an \ct{ImageView} is created with the picture of an arrow as background. It is located in a \ct{LinearLayout} which is placed in the hierarchy at same level as the \ct{ScrollView}. The \ct{ImageView} has its visibility set to \textit{invisible}. This can be seen in listing \ref{lst:arrowXML}.

\begin{lstlisting} [caption={A section of the xml-code in landscape$\_$mode}, label={lst:arrowXML}]
<LinearLayout
    android:id="@+id/land_arrow_linlay"
    android:orientation="horizontal"
    android:layout_width="wrap_content"
    android:layout_height="wrap_content">

    <ImageView
        android:id="@+id/arrow"
        android:layout_width="match_parent"
        android:layout_height="match_parent"
        android:background="@drawable/arrowtemp"
        android:visibility="invisible"/>
                
</LinearLayout>
\end{lstlisting}

We then created a method in the \ct{HorizontallyScroll} class called \ct{setupMarker}. This method aligns the marker above the first pictogram in the sequence, and sets its visibility to \textit{visible}. The alignment is done by first targeting the \ct{LinearLayout} which contains the \ct{ImageView} with the arrow. We then assign as much space as a pictogram for width and \ct{wrap$\_$content} for height, and define these as the layout parameters for this \ct{LinearLayout}.

We then add alignment rules for the \ct{LinearLayout}. This is done because it is placed inside a \ct{RelativeLayout}. We use \ct{ALIGN$\_$LEFT, mainLayout.getChildAt(0).getId()} and \newline \ct{ABOVE, mainLayout.getChildAt(0).getId()} to make the arrow stay above the first pictogram in the sequence. Last we center everything inside the \ct{LinearLayout}.

The arrow is set to \textit{visible}, and the \ct{ScrollView} is told to stay below the arrow with \ct{BELOW, arrow$\_$linlay.getId()}. The code can be seen in \ref{lst:markerMethod}.

\begin{lstlisting} [caption={Method for setting up the marker}, label={lst:markerMethod}]
private void setupMarker(){

    arrow_linlay = (LinearLayout) landscape.findViewById(R.id.land_arrow_linlay);
    RelativeLayout.LayoutParams myParams = new RelativeLayout.LayoutParams(getWidth() / numberOfVisibiblePictograms, LinearLayout.LayoutParams.WRAP_CONTENT);
    myParams.addRule(RelativeLayout.ALIGN_LEFT, mainLayout.getChildAt(0).getId());
    myParams.addRule(RelativeLayout.ABOVE, mainLayout.getChildAt(0).getId());
    arrow_linlay.setGravity(Gravity.CENTER);
    arrow_linlay.setLayoutParams(myParams);


    ImageView arrow = (ImageView) landscape.findViewById(R.id.arrow);
    arrow.setVisibility(VISIBLE);
    arrow.setTag("This_is_an_arrow_marker");

    RelativeLayout.LayoutParams myMain = (RelativeLayout.LayoutParams)this.getLayoutParams();
    myMain.addRule(RelativeLayout.BELOW, arrow_linlay.getId());
    this.setLayoutParams(myMain);
}
\end{lstlisting}

The marker can be seen attached to a five pictogram long sequence shown in figure \ref{fig:markerpic}, which is displayed in landscape-mode.
\begin{figure} [h!]
\centering
\includegraphics[width=0.9\textwidth]{Pics/Sprint3/landscape5picsCROP.png}
\caption{This is the arrow-marker displayed above the first pictogram in a sequence displayed in landscape-mode}
\label{fig:markerpic}
\end{figure}