\begin{longtable} { | c | p{12cm} | c | } 
\hline
	ID 	&	Issues	&		 Es. hours \\\hline
	4 	&	Destroy Sekvens if minimized	&	8 hours \\\hline
\caption{Issue ID 4}
\label{tab:spr3_homebuttonshoulddestroy}
\end{longtable}

A customer request was that if the application is ever minimized, whenever the user attempts to launch it again, it should open a fresh instance rather than returning to where the customer last left it. This turned out to be a more difficult task than assumed. An attempt was made in the first sprint to solve the issue, but the suggested idea was rejected. The suggested solution was to override Android's \ct{onStop} to also call \ct{onDestroy}. This would definitely fix the issue, as whenever the activity was not in focus, it would be destroyed. It however created the unintended effect that once opening another activity (SequenceActivity, PictoSearch), the original activity, now in the background would have been destroyed. This meant there was nothing to return to, once done with the opened activities. The issue was postponed, as the group did not see it as a major problematic issue.

The request was re-opened in sprint 3. At this point an attempt was made to override the tablet's Home and Back buttons to kill the application if pressed. It was however discovered that the Home button was atomic. It was neither possible to interrupt it or even set a flag to indicate if pressed. Instead a new variable was introduced to all activities of Sekvens, called \ct{assumeMinimize}. This was a Boolean, set to true by default. In the \ct{onStop} method, it would then be checked. If set to true, the application should be killed. This task included fetching the instances of any other open activities, and killing them as well, before killing the instance in which \ct{onStop} was run.

The catch to make this work, was to set the variable to false whenever a new Activity was opened. In this case, when \ct{onStop} was run, the application would not be killed, but \ct{assumeMinimize} would be reset to true. This behavior was copied to all Activities. The \ct{onStop} code to make \ct{assumeMinimize} work for MainActivity can be seen in \ref{lst:assumeMinimize}. If MainActivity was launced in \ct{nestedMode} where the guardian can insert a sequence into another, this means that both SequenceActivity and another instance of MainActivity is running in the background. They are then both killed before the activity itself, where \ct{onStop} was run, is killed.

\begin{lstlisting}[caption={onStop with the check for assumeMinimize}, label={lst:assumeMinimize}]
protected void onStop() {
    if (assumeMinimize) {
        //If in NestedMode, kill all open Activities. If not Nested, only this Activity needs to be killed
        if (nestedMode) {
            SequenceActivity.activityToKill.finish();
            MainActivity.activityToKill.finish();
        }
            finishActivity();
    } else {
        //If assumeMinimize was false, reset it to true
        assumeMinimize = true;
    }
    super.onStop();
}
\end{lstlisting}