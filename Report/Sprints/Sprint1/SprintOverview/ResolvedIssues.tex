\subsection{Resolved Issues} \label{subsec:spr1_resolved_issues}

At the end of sprint 1, the following issues from the sprint backlog were resolved.
\begin{longtable} { | c | p{12cm} | c | } 
\hline
	ID 	&	Issue	&		 Es. hours \\\hline
	6	& 	Changing application name	&	1 hours	\\\hline
\caption{Issue ID 6}
\label{tab:spr1_issue6}
\end{longtable}
This was a trivial task of changing the application name in the AndroidManifest of the project. The project is as mentioned now called `Sekvens' as opposed to the former name `Zebra'.

\begin{longtable} { | c | p{12cm} | c | } 
\hline
	ID 	&	Issue	&		 Es. hours \\\hline
	10	& 	Changing a sequence name is not intuitive	&	8 hours	\\\hline
\caption{Issue ID 10}
\label{tab:spr1_issue10}
\end{longtable}
Changing the name of a sequence was previously done by clicking on the title text, but it was not obvious that it was in fact editable (for reference see picture \ref{fig:Old_editName}).  The issue was first solved by adding a button which programatically places the cursor in the textfield and popping up the virtual keyboard. We felt like this cluttered the entire UI a little bit with too many buttons, so we decided to figure out another way to make it more apparent. We chose to go with a button-less approach, where we put a border on the title, and recoloring the background to white. The thought behind this was the user most likely associate black text on white background as being editable, as the most popular text editing programs today use that standard (the result of the solution is displayed at figure \ref{fig:New_editName}). \note{change picture to the real deal}

\begin{figure} [h!]
\centering
\begin{minipage}{.7\textwidth}
\centering
\includegraphics{Pics/Sprint1/Gammelt/change_sequence_name}
\caption{A picture of the editable name before}
\label{fig:Old_editName}
\includegraphics{Pics/Sprint1/Gammelt/change_sequence_name}
\caption{A picture of the editable name after}
\label{fig:New_editName}
\end{minipage}\hfill
\end{figure}

\begin{longtable} { | c | p{12cm} | c | } 
\hline
	ID 	&	Issue	&		 Es. hours \\\hline
	11	& 	Analyzing TODOs in code	&	16 hours	\\\hline
	17	& 	Solve TODO in SequenceViewGroup - Can child be bigger than parent?	&	2 hours	\\\hline
\caption{Issue ID 11 and 17}
\label{tab:spr1_issues11_17}
\end{longtable}

Solving TODO's of the previous team who worked on the project was one of the more difficult tasks. They were usually very unclear, examples being "TODO ???" and "this is ridiculous". With statements being that vague and seemingly not meant for other teams to understand, made this a time expensive task. The estimate of 16 hours however fit surprisingly well for both analyzing and solving the issues. This is partly due to some of the TODO's seemingly representing a lack of understanding for the code. These TODO's could thus just be removed. Issue 17 was created as a seperate task because it was difficult to troubleshoot, but it was found to that code already existed to ensure that a child could not be bigger than a parent.

\begin{longtable} { | c | p{12cm} | c | } 
\hline
	ID 	&	Issue	&		 Es. hours \\\hline
	14	& 	Import temporary pictures into Sequence	&	4 hours	\\\hline
\caption{Issue ID 14}
\label{tab:spr1_issue14}
\end{longtable}
While Sequence was runnable from the beginning of the sprint, it had no pictograms available in anyway. While there at the end of the sprint still was no database connection, having some local images to use in the application was desired. It was discovered that one of the previous groups had hardcoded paths and names for pictograms that we had no access to. Pasting new images with matching titles on the hardcoded path (The SD card of the device) allowed us to have images to work with in the application.

\begin{longtable} { | c | p{12cm} | c | } 
\hline
	ID 	&	Issue	&		 Es. hours \\\hline
	18	& 	Child selected upon opening the program is not highlighted	&	2 hours	\\\hline
\caption{Issue ID 18}
\label{tab:spr1_issue18}
\end{longtable}
This issue was looked into but turned out to be difficult to solve. However, while it was in fact never fixed, it also became a non-issue due to new requirements. These specified that the child list where the issue occurred should not be available from within Sequence. Retrieving childs should be done from the launcher application. The issue is therefore closed.

\begin{longtable} { | c | p{12cm} | c | } 
\hline
	ID 	&	Issues	&		 Es. hours \\\hline
	19	& 	Zebra is installed with two app-icons	&	16 hours	\\\hline
\caption{Issue ID 19}
\label{tab:spr1_issue19}
\end{longtable}
Debugging why the application was installed twice upon compiling was another hard task. This was one of the first issues we attempted to solve, and lack of knowledge about Android made this simple fix a time-wise underestimated task. We did not immediately realize that the issue was not within the Sequence project, but one of its dependencies. Furthermore, another group was working simultaneously on the same issue, which was not discovered in the process. This happened because the time was spent trying to understand the issue. At that point it was a simple fix, and it was not before then it could be discovered that it was the same issue the other group was trying to solve.

\begin{longtable} { | c | p{12cm} | c | } 
\hline
	ID 	&	Issue	&		 Es. hours \\\hline
	20	& 	When a sequences is created it should have a return to overview button	&	8 hours	\\\hline
\caption{Issue ID 20}
\label{tab:spr1_issue20}
\end{longtable}
When displaying a created sequence it was not apparent how the user was able to return to the overview. The former way was pressing the `back' on the android tablet itself, but it seemed rather unintuitive as you were supposed to go to the Overview, and not back to editing the sequence. This was solved by adding a button to allow the user to return by pressing it - and furthermore overriding the hardware button to return to Overivew (A picture of the button is displayed at figure \ref{fig:Old_backButton}).
\begin{figure} [h!]
\centering
\includegraphics{Pics/Sprint1/Gammelt/back_button}
\caption{A picture of a simple button in `Sekvens'}
\label{fig:Old_backButton}
\end{figure} 