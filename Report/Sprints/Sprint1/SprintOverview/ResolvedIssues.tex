\subsection{Resolved Issues} \label{subsec:spr1_resolved_issues}

At the end of sprint 1, the following issues from the sprint backlog were resolved.
\begin{itemize}
\item \#6: This was a trivial task of changing the application name in the AndroidManifest of the project.
\item \#10: Changing the name of a sequence was previously done by clicking on the title text, but it was not obvious that it was in fact editable. The issue was solved by adding a button which programatically places the cursor in the textfield and popping up the virtual keyboard.
\item \#11: Solving TODO's of the previous team who worked on the project was one of the more difficult tasks. They were usually very unclear, examples being "TODO ???" and "this is ridiculous". With statements being that vague and seemingly not meant for other teams to understand, made this a time expensive task. The estimate of 16 hours however fit surprisingly well for both analyzing and solving the issues. This is partly due to some of the TODO's seemingly representing a lack of understanding for the code. These TODO's could thus just be removed.
\item \#14: While Sequence was runnable from the beginning of the sprint, it had no pictograms available in anyway. While there at the end of the sprint still was no database connection, having some local images to use in the application was desired. It was discovered that one of the previous groups had hardcoded paths and names for pictograms that we had no access to. Pasting new images with matching titles on the hardcoded path (The SD card of the device) allowed us to have images to work with in the application.
\item \#19: Debugging why the application was installed twice upon compiling was another hard task. (\#19) This was one of the first issues we attempted to solve, and lack of knowledge about Android made this simple fix a time-wise underestimated task. We did not immediately realize that the issue was not within the Sequence project, but one of its dependencies. Furthermore, another group was working simultaneously on the same issue, which was not discovered in the process. This happened because the time was spent trying to understand the issue. At that point it was a simple fix, and it was not before then it could be discovered that it was the same issue the other group was trying to solve.
\item \#20: When displaying a created sequence it was not apparent how the user was able to return to the overview. This was solved by adding a button to allow the user to return by pressing the button.
\end{itemize}