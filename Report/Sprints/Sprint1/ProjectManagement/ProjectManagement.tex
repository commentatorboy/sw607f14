\section{Project Management}\label{spr1_projmana}
To manage the GIRAF multi-project it was chosen to use a variant of Scrum. The version divided the semester into sprints with different focuses on each sprint (such as refactoring in sprint 1). At the end of each sprint, each group would present their work and issues to the customers and other groups. Furthermore we would start planning the next sprint. We used an issue-tracker called Redmine to manage the projects within the GIRAF multi-project, along with a Jenkins builder to monitor compile errors. \\
We also decided to use Scrum to manage the development of Sekvens. This meant that we held daily stand-up meetings to discuss what everyone was working on, how it was going and whether they needed help from the group to solve their issues. Scrum uses a product backlog to track issues for the product. Scrum also uses a sprint backlog, which contains the issues planned for the current sprint. We also used an internal group issue-tracker called AgileWrap, to keep track of the issues in the multi-project but also additional issues regarding research and so on. The issues from AgileWrap are not listed in this report \cite{agilewrap}.