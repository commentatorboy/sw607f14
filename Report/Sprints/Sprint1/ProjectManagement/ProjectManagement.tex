\section{Project Management}\label{spr1_projmana}
We decided as a whole to use a version of Scrum to manage the entirety of the GIRAF. The version devided the semester into sprints with different focuses on each sprint (such as refactoring the first sprint). In the end of each sprint, each group would present their work and issues throughout the sprint to the customers and other groups -furthermore start planning the next sprint. We used an issue-tracker called Redmine to manage this for the multi-project, along with a Jenkins builder to supply an overview of the project as a whole. \\
We also decided to use Scrum to manage our own little project. This meant that we held daily stand-up meetings to discuss what everyone was working on, how it was going and whether they needed something from the group to solve their assignment. Scrum requires a product backlog to record the requirements to the product, those were decomposed into small issues. Scrum also requires a sprint backlog, which held the issues planned for the current sprint We also used an internal group issue-tracker called AgileWrap, to keep track of the issues in the multi-project but also additional issues regarding research and so on. The issues from AgileWrap are not listed in this report. 
We managed our sprint backlog for the first sprint ourselves, and for each issue recorded in the issue tracker we estimated the number of hours it would take to resolve the issue, and gave it a priority.