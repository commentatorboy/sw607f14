\begin{longtable} { | c | p{12cm} | c | } 
\hline
	ID 	&	Issues	&		 Es. hours \\\hline
	32 	&	Display some temporary sequences in overview	&	4 hours \\\hline
\caption{Issue ID 32}
\label{tab:spr2_fakesequences}
\end{longtable}

This issue was made because the LocalDB was not working. Not having any sequences in the application caused a problem where we were not able to access some parts. Since that the application has its own model of Sequences, it was possible to resolve this by creating temporary sequences to display while working with the application. The code for creating fake sequences is shown in \ref{lst:fakeseq}

\begin{lstlisting} [caption={The code in the example the code used to create fake sequences}, label={lst:fakeseq}]

	private void loadSequences() {
        //TODO createFakeSequences is a temporary fix to generate some Sequences
	    List<Sequence> list; // = SequenceFileStore.getSequences(this, selectedChild);
		list = createFakeSequences();
        selectedChild.setSequences(list);
	}

    private List<Sequence> createFakeSequences() {

        Sequence s = new Sequence();
        s.setTitle("TEST SEQUENCE");
        s.setImageId(10);
        s.setSequenceId(5);

        Pictogram a = new Pictogram();
        Pictogram b = new Pictogram();
        Pictogram c = new Pictogram();
        a.setPictogramId(0);
        b.setPictogramId(1);
        c.setPictogramId(2);
        s.addPictogramAtEnd(a);
        s.addPictogramAtEnd(b);
        s.addPictogramAtEnd(c);

        List <Sequence> list = sequences;
        for (int i = 0; i < 12; i++) {
            list.add(s);
        }
        return list;
    }

\end{lstlisting}