\section{Sprint Evaluation}\label{sec:spr2_spreval}
In this sprint we collaborated with the Livshistorier-, the iOS-, and the PiktoOplæser groups in order to structure a new database model (see section \ref{sec:database_planning}) and negotiate the terms of the new project `Sequenceviewer' (see section \ref{sec:sequenceviewer}).
We were assigned as the group to work on Sequenceviewer, which would be a general template for showing sequences.
The OasisLib group did make some minor changes to their functionality which meant we had to adapt Sekvens to it. This meant we had to spend time adjusting (see section \ref{sec:adhoc}).

We made many cosmetic changes and added a few new functionalities. These can be seen in section \ref{subsec:spr2_resolved_issues} about resolved issues.
In the sprint-end presentation the stakeholder questioned about the functionality for making a choice between pictograms and making nested sequences.
Therefore we had to specify what we actually meant with these functions.
Another important thing was to make sure that the we did not show any half pictograms -meaning we have to create some sort of snapping to make sure that half a pictogram is never shown.

All groups received their official requirements and translated them into issues for this sprint, which can be seen in appendix \ref{app:reqgroup1}. We also wrote a report to the requirement group, because we wanted some specific feedback. The questions can be seen in appendix \ref{app:reqquestion}, and the answer from the requirements group can be seen in appendix \ref{app:reqanswer}.