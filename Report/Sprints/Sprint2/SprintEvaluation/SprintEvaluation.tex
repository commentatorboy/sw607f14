\section{Sprint Evaluation}\label{sec:spr2_spreval}
In this sprint we collaborated with the `Livs Historier', the iOS group and `PiktoOplæser' in order to structure a new database model (see section \ref{sec:database_planning}) and negotiate the terms of the new project `Sequenceviewer' (see section \ref{sec:sequenceviewer}).
We were assigned as the group which works on `Sequenceviewer', which should be a general template for showing a sequence.
The `Oasislib' group did made some minor changes to their functionality which meant we had to adapt the `Sekvens' application to it. This meant we had to spend hours adjusting (see section \ref{sec:adhoc}).


We made many cosmetic changes and added a few new functionalities. These can be seen in \ref{subsec:spr2_resolved_issues}.
In the sprint-end demonstration the stakeholder questioned about the functionality for making a choice between pictograms and making nested sequences.
Therefore we have to specify what we actually mean with these functions.
Another important thing, is to make sure that the pictograms must not show only a part of its picture -meaning we have to create some sort of snapping to make sure that half a pictogram is never shown.


All groups got their official requirements and translated them into issues for this sprint. We wrote a specific report to the requirement group, because we wanted some very specific feedback (see appendix \note{indsæt reference til Appendix})