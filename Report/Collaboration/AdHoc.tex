\section{Other Ad-hoc Activities}\label{sec:adhoc}
Apart from the the major collaborations, Sekvens was in a unique position where we had a lot of dependencies. As a result, our work often overlapped with the work of other groups. Below is described ad-hoc activities that took place on a day-to-day basis.

\paragraph{GIRAF\_Components}
Since it was decided to share common GUI components, most graphical components in Sekvens had to be changed. Using the components from the GIRAF\_Components group was made very intuitive, and we were mostly able to use them without problem. Sometimes however, we could not resolve either how to use a component properly, or how to give a component a certain functionality. This resulted in several cases where we lent one of their team members to provide assistance, or alternatively cases where new functionality had to be added to the GIRAF\_Components.  Furthermore, since this group did not use its own components, other than for testing, a few errors was caught via the Sekvens application, and corrected by the GIRAF\_Components group.

\paragraph{OasisLib}
Ad-hoc activities were also done with the OasisLib, assisting in both using their classes properly and providing new functionality on request. Given that Sekvens uses OasisLib for everything related to sequences, these run-ins occured often. As with the GIRAF\_Components, our position meant that several bugs were discovered and later corrected by OasisLib.

\paragraph{Launcher}

\paragraph{PictoSearch}

\paragraph{Requirements}