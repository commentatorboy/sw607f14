\chapter{Introduction}\label{chap:introduction}
The proposed project for this semester is a multi-project involving the entire Software 6 class. The overall purpose of the project is to create a set of applications for the Android platform, designed to help autistic people and their guardians throughout their day. Examples of this are the application "Stemmespillet" which helps people learn how loud they are, and "Piktooplæser" which helps with communication by using pictures and audio.
Previous Software 6 classes have worked on the same multi-project. This means that even from the beginning of the semester there was an existing code base to work from. While previous classes had developed several applications, none of them was in a state where they were ready to release. In general, most applications had crash issues and were missing a required connection to the designed database. Some applications were missing features or had implemented some bad workarounds to make it run. It was decided that the focus of this class should be to refactor and rework these programs to a degree where they could be released. Therefore, no new applications was made this semester.
The development of applications has been based on requests and feedback from customers consisting of guardians working with autistic people. This means that at the end of the semester, the applications ideally consist of tools that both the guardians and autistic people can use effectively.