\section{Reflections}\label{sec:reflections}

The customers, possible end users and collaboration across groups made this semester very interesting. It was very different from anything we had tried before, but the feeling is that it overall went well. Furthermore, this was the first time we had had a large code-base at the beginning of a semester. In previous semesters, any code had been created from scratch. It was interesting to try gaining an overview of other students code, learning exactly what and how code was done.

Since this is the first time we have collaborated across groups, there was certain unexpected issues with management. At the beginning of sprint 1, we had little idea how many dependencies there were for Sekvens, even if it was not connected properly to all of them. Dependencies, and the work associated with managing them took much effort. This sometimes allowed for less time to be spent on Sekvens itself. Under the circumstances, it went quite well. But if we were to do it again, it would have been ideal to create a debugging mode from the start to include some hardcoded data (Persons and sequences). This could have saved some time, when stuck because of a failing dependency. Given the lack of experience with multi-projects, we were however unable to predict it.

Sharing one tablet across two groups also had unforeseen problems. During the problems it would vary how far ahead we were. This could in the worst case mean that the two applications had dependencies to different versions of the database. It was sometimes possible to access two tablets, but the best solution had been to have at least one per group. This was also due to slow emulators which could take as long as 5 minutes to start up.

The overall experience with this semester has been good. As a group we have become better at both programming and management across groups.