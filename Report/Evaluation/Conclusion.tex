\section{Conclusion}\label{sec:conclusion}

At the beginning of the semester, some code patterns was decided upon (See section \ref{spr1_design_patterns}). These patterns were followed throughout the project with one notable exception. MainActivity is no longer singleton, in that opening \ct{nestedMode} is a second instance. When deciding on code patterns it was not considered that having a second instance for a different purpose could be relevant. The intent with the singleton pattern was rather that several instances of the application should not exist at the same time. Given the code re-use that MainActivity in \ct{nestedMode} gave grounds for, we find that breaking the code pattern is justified, especially given that it does not duplicate a running activity.

The goal of this semester was to finish the existing application rather than develop new ones. As described through the report we have continued the development on Sekvens, pushing for a state where it would be ready for a potential release of GIRAF. As it has been described from sprint 3 and forward, Sekvens was split up into a management application and the hidden SequenceViewer.

The latter is in a state where it is functional, but missing some polishing. It also lacks the feature of somehow pointing to the current pictogram in a sequence. The request for this feature is known from the work of the requirements group. During the usability tests we however got surprisingly little feedback regarding SequenceViewer. In fact there was only a side-question of whether it was possible to see the sequence vertically rather than horizontally, which was later seen in settings. The issue of missing indicators for choices was also ignored during the tests, another feature we have through requirements.

In other words, judging on the usability test alone, SequenceViewer could be done. Whether the two usability tests gives a clear image of whether it is perfect is doubtable. Looking aside from the missing indicators for choices and the pointer to pictograms, we conclude that more testing is required to give a good evaluation on its state. The indication is however that while it is not finished, it is headed in the right direction.

Sekvens had much groundwork done when we started on the project. This allowed us to develop requested features and polish it. At the usability tests, two suggestions were made; that we use a question mark to indicate choices and the ability to send a sequence by email. The latter was implemented with help from the Livshistorier group as described in \ref{sec:adhoc}. Indicators were however not done in time.
From the usability tests we gathered that while some features were problematic to find, it was possible, even for a beginner to use the application. This is very positive, and indicates that the design is intuitive.
It was not ideal to have the usability tests in the last sprint as this gave very little room to develop any requests. The fact that only one request could not be developed in time could be an indicator that Sekvens is in a good state. This does however not mean that it is ready for release. We find that the missing indicators of choices and nested sequences is something that should not be left out. Furthermore, there are several minor issues that could be improved upon, as described in \ref{sec:future_work}. If GIRAF will be developed on for another semester, we would recommend that Sekvens be worked on for 1-2 months.

Lastly, it should be emphasized that two usability tests can not give a clear answer as to the state of the projects. Further usability testing would be required for a full analysis of the projects. What can be concluded is that functionality- and feature-wise, Sekvens is close to a release version.