\section{Future Work}\label{sec:future_work}

In the following section we will discuss the future work of Sekvens. Aside from the big functional improvements listed below, there are a lot of optimizing and improvements to the general flow of the project. The general flow could be improved through more usability testing and applying whatever improvements they suggest - as to missing dialog boxes, misinterpreted buttons or unclear instructions.

\paragraph{Nested Sequences} as mentioned at the section nested sequences (see \ref{tab:spr4_nestedSequence}), there is a couple of different issues with the implementation chosen. As mentioned the biggest problem seem to be that it is possible to alter/delete sequences as the main sequence simply references the nested sequence. This is a problem because deleting the nested sequence will at first make the application crash, but later when a new sequence is made with the same ID as the deleted nested sequence used to have, link to a wrong sequence. There is various different ways to resolve this issue, and one of them could be creating a new `hidden' sequence which is only linked to through the main sequence. The downside of this solution is the amount of memory used to store `hidden' sequences which in themselves offer nothing to the application.

The other problem with nested sequences is the possibility to create circular nesting, meaning sequence $\#1$ can be nested into sequence $\#2$ while sequence $\#2$ is nested into sequence $\#1$. This is no really a problem while running the app, seeing it does not crash the guardian version of the application. The problem occurs when the child is supposed to watch the sequence. Keep in mind that sequenceviewer does not fully support nested sequences yet, but the way we intended sequenceviewer to open nested sequences was to open a new instance of sequenceviewer once the nested sequence is reached. This way it would be possible to support forever nested sequences -and not limiting the application. If sequenceviewer was implemented like this, circular nesting would cause the application to keep calling itself every time the nested sequence was reached thus never ending. Somehow the application must prevent circular nesting, and the easiest way would probably be to when \ct{setupNestedMode} is called through \ct{sequenceActivity}, and a sequence is chosen you could save the \ct{sequenceId} in a temporary list of \ct{Integers}. Then the next time \ct{setupNestedMode} is called, you could exclude all of the sequenceIds already in the \ct{List}.

\paragraph{Copy and delete dialog boxes} The functionality is already completely there, it is possible to both copy and delete sequences. The problem is the \ct{GGridView} provided by `GComponent' group along with the \ct{PictogramView} created in `Sekvens' takes a lot of memory. This results in when you mark a lot of sequences it takes way to much memory and often run out of memory. This is quite a big problem as the application crashes. There is not really an easy solution to this, the solutions are either optimize \ct{GGridView} and \ct{PictogramView} or alternatively figure out a less memory dependent method to copying or deleting sequences. 

\paragraph{Snapping Scrollviews} This problem is a bit more coding dependent. It was a requirement listed by the costumers \note{Indsaet reference til requirements analysis} that the children were never exposed to half pictograms as they might be confusing. This means every place that has \ct{PictogramViews}, needs to implement a snapping grid ensuring that only full pictograms are shown. This is the case in the overviews \ct{GGridView} and the \ct{SequenceViewGroup} in \ct{sequenceActivity}. In \ct{sequenceActivity} it might be pretty arbitrary to do as it always shows 4 \ct{PictogramViews} horizontally in 1 row and scale them according to the width of the screen. It might be harder in the \ct{MainActivity} as it scales according to width but stacks rows as sequences are added. 

\paragraph{Implement Sequenceviewer with PiktoOplaeser}
During this semester, PiktoOplaeser did not manage to implement the database functionality finished during sprint 4. Therefore, Sequenceviewer did not have the possibility to implement their way of using the database, and setting up functionality for their application. Next semester, cooperation should finish this absence of functionality. From the Sequenceviewer perspective, code should be written for their database-usage, and provide the PiktoOplaeser-group with information and help with setting up the intents and call to Sequenceviewer.

\paragraph{Miscellaneous}
The following is a list of minor changes which does not take long to fix, but was big enough to receive a notable mention. 

\begin{itemize}
\item \ct{isInEditMode} is an ancient relic of the past, a variable that used to be from times when \ct{sequenceActivity} was used to both display and edit sequences. This boolean is completely unnecessary now, and could be removed from the entire project.
\item When entering Sekvens without a \ct{childId} being sent from launcher, the \ct{GProfileSelector} pops up and you need to pick a child. It is right now possible to press the hardware back button. The problem is that the Profile Selector is a view and not an activity, so you cant simply override \ct{OnBackPressed}.
\item When copying sequences there is no confirmation that the sequences have been copied. This could be achieved through a dialog box or simply closing the entire window completely.
\item When altering settings, there is no save button. It simply saves every time back is pressed, giving no confirmation the settings are saved. This is a simple problem which could be solved through a save button or a confirmation dialog box when pressing back.
\item The exit button needs to be invisible in \ct{NestedMode}. 
\item Indicators for choices and nested sequences, when editing and showing a sequence. This could be achieved through a mark on the \ct{PictogramView}.
\item Right now the \ct{GGridView} in \ct{MainActivity} always shows 5 \ct{PictogramViews}. This should be dynamic such that the amount of sequences shown are determined by the width of the screen.
\end{itemize}

\subsection{Rejected tasks in Product Backlog}
Over time, the Product Backlog was cluttered due to tasks that had for various reasons was invalidated. This could happen due to new issues, new knowledge about an issue or change of focus. Therefore the backlog was looked through, and the following issues has been removed from it.

\begin{longtable} { | c | p{12cm} | c | } 
\hline
	ID 	&	Issues	&	 Es. hours \\\hline
	7	& 	Marked pictogram not highlighted		& 	8 hours  \\\hline
	8	& 	Child mode needs to be more intuitive		& 	32 hours  \\\hline
	9	& 	Sequence does not look clickable 	& 	16 hours  \\\hline
	22	& 	Portrait mode	& 	32 hours  \\\hline
\caption{This is a list rejected after reevaluation}
\label{tab:spr4_sw_prodblog}
\end{longtable}


\subsection{Remaining Product Backlog}
Even after sorting through the backlog at the end of the last sprint, some issues still remained. This was both tasks that have never been picked up and tasks not completed in sprints. The following is the final product backlog. This can be used as reference for anyone wanting to continue development on Sekvens.

\begin{longtable} { | c | p{12cm} | c | } 
\hline
	ID 	&	Issues	&	 Es. hours \\\hline
	2	& 	Updating a picture in sequence -$>$; no update in overview	& 	8 hours  \\\hline
	13	& 	Resizing delete buttons on pictograms	& 	4 hours  \\\hline
	16	& 	It is possible to make identical sequences for the same child	& 	8 hours  \\\hline
	23	& 	SequenceGrid needs to be dynamic	& 	8 hours  \\\hline
	36	& 	Create snap in SequenceActivity	& 	4 hours  \\\hline
	37	& 	Create snap in SequenceGrid	& 	4 hours  \\\hline
	46	& 	improve user handling 	& 	4 hours  \\\hline
	49	& 	Sequenceviewer: highlighting with a outline 	& 	16 hours  \\\hline
	51	& 	Sequenceviewer: in settings dialog - choose what type of highlighting there should be	& 	2 hours  \\\hline
	53	& 	Sequenceviewer: blur pictures out of focus	& 	8 hours  \\\hline
	63	& 	Sequenceviewer: Display nested sequences 	& 	16 hours  \\\hline
	69	& 	Sequenceviewer: Marker drag over to next pictogram 	& 	16 hours  \\\hline
	70	& 	Sequenceviewer: Marker stays at pictogram when scrolling the sequence 	& 	8 hours  \\\hline
	71	& 	Sequenceviewer: Marker snaps to pictogram 	& 	16 hours  \\\hline
	72	& 	Sequenceviewer: 2 rows/columns. 1 with a marker, 1 with a sequence 	& 	16 hours  \\\hline
	73	& 	Sequenceviewer: display 2 rows/columns 	& 	16 hours  \\\hline
	74	& 	Sequenceviewer: While playing sound, follow the pictogram which has sound being played 	& 	32 hours  \\\hline
\caption{This is a list remaining for Sekvens and sequenceviewer}
\label{tab:spr4_sw_prodblog}
\end{longtable}


