\chapter{Resume of report}\label{app:resume}
For the last few years, Software 6 studens have worked on a series of applications under the common project name GIRAF. These applications are designed to help individuals suffering from autistic spectrum disorder and the ones who help take care of them.

Because persons suffering from autistic spectrum disorder often have significant disabilities, being a caretaker is not always an easy task. Common disabilities are the lack of ability to communicate. Some will for example have a very small vocabulary if any at all. This is an obvious hindrance both for the affected persons but also the people they interact with.

GIRAF puts focus on some of the difficulties and attempts to provide tools to help overcome them. One of the applications within GIRAF is Sekvens. This application is designed for caretakers to create sequences of pictograms in order to help guide a person with an autism disability through otherwise difficult tasks. Every person could then have their own unique list of sequences, created by their caretaker(s).

Depending on the degree of handicap, a sequence could be as simple as a series of pictograms showing the steps required to wash hands properly.

With Sekvens being an existing application from previous Software 6 semesters, a codebase existed at the beginning of our semester. It was graphically able to create and edit sequences with no additional features. It was however also very close to be a stand-alone application as it was not tied into GIRAF. It was for example not set up to cooperate with other GIRAF projects such as the database and the Launcher which is where applications were meant to be launched from.

For this semester, the goal has been to finish Sekvens and tie it into GIRAF the way it was meant to. This was done, using the feedback of customers as guidelines. Throughout the semester Sekvens had its codebase reduced, although preserving functionality. This was a consequence of tying it into other projects, depending on others code. A lot of new features were designed on requests of customers. These includes management tools to for example copy M sequences to N persons. Sequences themselves has also been advanced to include new features such as letting the user create choices for the person suffering from an autism spectrum disorder.

Sekvens was taken to a near-complete state where almost all customer requests had been addressed. The report covers this process in detail.

At the same time, a second project was started half-way into the semester. A few other projects were also supposed to display sequence-like series of pictograms. For this reason development was initiated on a Sequenceviewer which ideally should be able to play sequences for anyone wanting to use it, while also customized for everyone, depending on their needs.

This project was however not finished. The report covers the Sequenceviewer from the initiation to the final state for this semester - a viewer with the core functionalities implemented, although still lacking a few requested features.